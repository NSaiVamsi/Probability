\documentclass{beamer}
\usetheme{CambridgeUS}
\usepackage{amsmath}

\def\inputGnumericTable{}
\setbeamertemplate{caption}[numbered]{}

\usepackage{enumitem}
\usepackage{amsmath}
\usepackage{amssymb}
\usepackage{gensymb}
\usepackage{graphicx}
\usepackage{txfonts}
\usepackage[latin1]{inputenc}
\usepackage{color}
\usepackage{array}
\usepackage{longtable}
\usepackage{calc}
\usepackage{multirow}
\usepackage{hhline}
\usepackage{ifthen}

\usepackage{lscape}




\title{Assignment 5 Probability}
\author{Narsupalli Sai Vamsi}
\date{June 2022}
\begin{document}
\begin{frame}
\titlepage
 \begin{abstract}
     This pdf consists the solution to the question 6.47 from in Papoulis pillai
 \end{abstract}   
\end{frame}
\begin{frame}{Outline}
\tableofcontents
\end{frame}
\section{Question 6.47}
\begin{frame}{Question 6.47}
\begin{block}{Q 6.47}
 The random variables $X_l$ and $X_2$ are jointly normal with zero mean. Show that their density can be written in the form.
 $$
 f(x_1,x_2) = \frac{1}{2\pi\sqrt{\Delta}}exp{\left\{-\frac{1}{2}XCX^{-1}\right\}}
 $$
 where C is given by $$ C =  \begin{bmatrix}
 \mu_{11} & \mu_{12} \\
 \mu_{21} & \mu_{22}
 \end{bmatrix}$$
 where $X:[x_1,x_2]$,$\mu_{ij} = E(x_ix_j)$ and $\Delta = \mu_{11}\mu_{22}- {\mu}^2_{12}$
\end{block}
\end{frame}
\section{Solution}
\begin{frame}{Solution}
\begin{block}{Solution}
\begin{align*}
 C= \begin{bmatrix}
{\sigma_1}^2 & r\sigma_1\sigma_2\\
r\sigma_1\sigma_2 & {\sigma_2}^2 
\end{bmatrix} && \Delta={\sigma_1}^2{\sigma_2}^2(1-{r}^2) 
\end{align*}
\begin{align*}
    C^{-1}= \begin{bmatrix}
    \frac{1}{(1-{r}^2){\sigma_1}^2} & \frac{r}{(1-{r}^2){\sigma_1}{\sigma_2}}\\
    \frac{r}{(1-{r}^2){\sigma_1}{\sigma_2}} & \frac{1}{(1-{r}^2){\sigma_2}^2}
    \end{bmatrix}
\end{align*}
We know that $X : [x_1,x_2]$\\
$$
XC^{-1}X^{t} = \frac{1}{(1-{r}^2)}\left(\frac{{x_1}^2}{{\sigma_1}^2} - 2r\frac{x_1x_2}{\sigma_1\sigma_2} + \frac{{x_2}^2}{{\sigma_2}^2}    \right)
$$   

\end{block}
\end{frame}
\end{document}

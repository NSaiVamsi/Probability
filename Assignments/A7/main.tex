\documentclass{beamer}
\usetheme{CambridgeUS}
\usepackage{amsmath}

\def\inputGnumericTable{}
\setbeamertemplate{caption}[numbered]{}

\usepackage{enumitem}
\usepackage{amsmath}
\usepackage{amssymb}
\usepackage{gensymb}
\usepackage{graphicx}
\usepackage{txfonts}
\usepackage[latin1]{inputenc}
\usepackage{color}
\usepackage{array}
\usepackage{longtable}
\usepackage{calc}
\usepackage{multirow}
\usepackage{hhline}
\usepackage{ifthen}

\usepackage{lscape}




\title{Assignment 7 Probability}
\author{Narsupalli Sai Vamsi}
\date{June 2022}
\begin{document}
\begin{frame}
\titlepage
 \begin{abstract}
     This pdf consists the solution to the question 12.4from in Papoulis pillai
 \end{abstract}   
\end{frame}
\begin{frame}{Outline}
\tableofcontents
\end{frame}
\section{Question 12.4}
\begin{frame}{Question 12.4}
\begin{block}{Q 12.4}
Show that the process $x(t) = a.e^{j(\omega t + \phi)}$ is not correlation-ergodic.
\end{block}
\end{frame}
\section{Solution}
\begin{frame}{Solution}
\begin{block}{Solution}
Given that $x(t) = a.e^{j(\omega t + \phi)}$ then the time average is given by
$$\frac{1}{2T}\int_{-T}^{T} x(t+\tau)x^{*}(t) = e^{j\omega\tau} a^2$$\\
As it goes to infinte but not zero it is not correlation-ergodic
\end{block}
\end{frame}
\end{document}

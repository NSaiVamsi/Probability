\documentclass{beamer}
\usetheme{CambridgeUS}
\usepackage{amsmath}

\def\inputGnumericTable{}
\setbeamertemplate{caption}[numbered]{}

\usepackage{enumitem}
\usepackage{amsmath}
\usepackage{amssymb}
\usepackage{gensymb}
\usepackage{graphicx}
\usepackage{txfonts}
\usepackage[latin1]{inputenc}
\usepackage{color}
\usepackage{array}
\usepackage{longtable}
\usepackage{calc}
\usepackage{multirow}
\usepackage{hhline}
\usepackage{ifthen}

\usepackage{lscape}




\title{Assignment 3 Probability}
\author{Narsupalli Sai Vamsi}
\date{May 2022}


\begin{document}

\begin{frame}
 \titlepage  
 \begin{abstract}
    This pdf consists the solution to the question 2.19 from in Papoulis pillai
\end{abstract}
\end{frame}
\begin{frame}{Outline}
\tableofcontents
\end{frame}
\section{Question2.19}
\begin{frame}{Question 2.19}
 (Q2.19) A box contains m white and n black balls. Suppose k balls are drawn. Find the probability of drawing at least one white ball.   
\end{frame}
\section{Solution}
\begin{frame}{Solution}
   \begin{block}{Solution:}
     Given that we have m white and n black balls and k balls are drawn from them and we have to find the probability of drawing at least 1 white ball.
     P(at least 1 white ball) = 1 - P(all the drawn k balls are black)
   \end{block}
\end{frame}
\begin{frame}{Solution}
\begin{block}{Solution:}
P(all the drawn k balls are black) = $\frac{\begin{pmatrix} n \\ k\end{pmatrix}}
{\begin{pmatrix}m+n\\k\end{pmatrix}}$\\
from this we can say P(at least 1 white ball) =  
1 - $\frac{\begin{pmatrix}n\\k\end{pmatrix}}{\begin{pmatrix}m+n\\k\end{pmatrix}}$
\end{block}
\end{frame}
\end{document}

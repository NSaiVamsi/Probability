\documentclass{beamer}
\usetheme{CambridgeUS}
\usepackage{amsmath}

\def\inputGnumericTable{}
\setbeamertemplate{caption}[numbered]{}

\usepackage{enumitem}
\usepackage{amsmath}
\usepackage{amssymb}
\usepackage{gensymb}
\usepackage{graphicx}
\usepackage{txfonts}
\usepackage[latin1]{inputenc}
\usepackage{color}
\usepackage{array}
\usepackage{longtable}
\usepackage{calc}
\usepackage{multirow}
\usepackage{hhline}
\usepackage{ifthen}
\providecommand{\cbrak}[1]{\ensuremath{\left(#1\right)}}
\usepackage{lscape}




\title{Assignment 9 Probability}
\author{Narsupalli Sai Vamsi}
\date{June 2022}
\begin{document}
\begin{frame}
\titlepage
 \begin{abstract}
     This pdf consists the solution to the question 15.15 from in Papoulis pillai
 \end{abstract}   
\end{frame}
\begin{frame}{Outline}
\tableofcontents
\end{frame}
\section{Question 15.15}
\begin{frame}{Question 15.15}
\begin{block}{Q 15.15}
Determine the mean time to absorption for the random walk model in Example 15-25. 
In the context of the gambler's ruin problem discussed there, show that the mean time to 
absorption for player A (starting with \$a) reduces to Eq. (3-53). 
\end{block}
\end{frame}
\section{Solution}
\begin{frame}{Solution}
\begin{block}{Solution}
The mean time to absorption satisfies (15-240). From there\\
$$m_i = 1+ \sum_{k\in T}^{}p_{ik}m_k = 1+ p_{i,i+1}m_{i+1}+p_{i,i-1}m_{i-1}$$\\
$$
m_i = 1 + pm_{i+1} +qm_{i-1}$$\\
replacing i with k we get\\
$$m_k = 1 + pm_{k+1} +qm_{k-1}$$\\
as p+q =1, we can write\\
$$p(m_{k+1}-m_{k})=q(m_k-m_{k-1}) - 1 $$
\end{block}
\end{frame}
\section{Solution}
\begin{frame}{Solution}
\begin{block}{Solution}
Let \\
$$ M_{k+1} = m_{k+1} - m_k$$
so now we get \\

$$  
M_{k+1} = \frac{q}{p}M_k - \frac{1}{p}\\
\implies M_{k+1}= \left({\frac{q}{p}}\right)^k M_1 - \frac{1}{p} \Bigg[1+ \left({\frac{q}{p}}\right)^k+ \left({\frac{q}{p}}\right)^k+ \dots +  \left({\frac{q}{p}}^{k-1}\right)\Big]
$$
\end{block}
\end{frame}
\section{Solution}
\begin{frame}{Solution}
\begin{block}{Solution}
From this we get\\ 
\begin{equation}
M_{k+1}&=
& \begin{cases}
   \left(\frac{q}{p}\right)^k M_1-\frac{1}{p-q}\left\{1-(\frac{q}{p})^k\right\}, &\text{$p \neq q$ }\\
   M_1 - \frac{k}{p}, & \text{$p = q$}
\end{cases}
\end{equation}
From this we get \\
$$m_k =\sum_{0}^{i-1} M_{k+1}$$
\begin{equation}
  m_k &=
  & \begin{cases}
     \left(M_1+ \frac{1}{p-q}\right)\sum_{k=0}^{i-1} \left(\frac{q}{p}\right)^k - \frac{i}{p-q}, & \text{$p \neq q$}\\
     iM_i - \frac{i(i-1)}{2p}, & \text{$p = q$}
    \end{cases}
\end{equation}
\end{block}
\end{frame}
\section{Solution}
\begin{frame}{Solution}
\begin{block}{Solution}
\begin{equation}
  m_k &=
  & \begin{cases}
     \left(M_1+ \frac{1}{p-q}\right)\ \left(\frac{1-(\frac{q}{p})^i}{1-(\frac{q}{p})}\right) - \frac{i}{p-q}, & \text{$p \neq q$}\\
     iM_i - \frac{i(i-1)}{2p}, & \text{$p = q$}
    \end{cases}
\end{equation}
where we have used $m_o = 0$.Similarly $m_{a+b} = 0$ gives
\begin{equation}
M_1 + \frac{1}{p-q} = \frac{a+b}{p-q}.\frac{1-(\frac{q}{p})^i}{1-(\frac{q}{p})^{a+b}}
\end{equation}
 \begin{equation}
 m_i &=
  &\begin{cases}
     \frac{a+b}{p-q}.\frac{1-(\frac{q}{p})^i}{1-(\frac{q}{p})^{a+b}} - \frac{i}{p-q},&\text{$p \neq q$}\\
     i(a+b-i),&\text{p = q}
   \end{cases}   
 \end{equation}   
\end{block}
\end{frame}
\section{Solution}
\begin{frame}{Solution}
\begin{block}{Solution}
Which gives for i = a\\
\begin{equation}
    m_a &=
    &\begin{cases}
    \frac{a+b}{p-q}.\frac{1-(\frac{q}{p})^a}{1-(\frac{q}{p})^{a+b}} - \frac{a}{p-q},&\text{$p \neq q$}\\
     ab,&\text{p = q}
     \end{cases}\\
    \centering = \begin{cases}
      \frac{a}{p-q} - \frac{a+b}{2p-1}.\frac{1-(\frac{p}{q})^a}{1-(\frac{q}{p})^{a+b}}  ,&\text{$p \neq q$}\\
     ab,&\text{p = q}
     \end{cases}
\end{equation}
by writing\\
$$
\frac{1-(\frac{q}{p})^a}{1-(\frac{q}{p})^{a+b}} =
1-\frac{(\frac{q}{p})^a -(\frac{q}{p})^{a+b}}{1-(\frac{q}{p})^{a+b}} =
1- \frac{1-(\frac{p}{q})^k}{1-(\frac{p}{q})^{a+b}}
$$
\end{block}
\end{frame}
\end{document}
